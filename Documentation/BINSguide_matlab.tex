\documentclass[12pt]{article}
\usepackage{graphicx}
\usepackage{float}
\usepackage{epstopdf}
\usepackage{amsmath}
\usepackage{amsfonts}
\usepackage{amssymb}
\usepackage{sidecap}
\usepackage{subfig}
\usepackage{ulem}
\usepackage{color}
\usepackage{placeins}
\usepackage[dvips,letterpaper,margin=0.5in]{geometry} %for one inch margins
\author{Lauren Goodfriend}
\title{Guide to the BINS educational solver}
\begin{document}
\maketitle
\tableofcontents

%----------------------------------------------------------------------
\section{Introduction}
%----------------------------------------------------------------------

The Berkeley Incompressible Navier-Stokes (BINS) solver is designed to be an educational tool for learning how to solve the two-dimensional incompressible Navier-Stokes equation with finite differences.  A gap exists between the basic numerical methods learned in a numerical partial differential equations class and the complexity of full Navier-Stokes research codes.  BINS is intended to form a bridge for engineering students who have some knowledge of numerical methods, but are unsure how to combine them into a working Navier-Stokes solver.  It may also be useful for students wrestling with an enormous research code for the first time, as a reference for the essential processes underlying most finite difference physical Navier-Stokes solvers.

%----------------------------------------------------------------------
\section{Getting started}
%----------------------------------------------------------------------

BINS is written in Matlab, so you will need a copy of Matlab to run the code.

The first step is to download the code tarball from \[ \texttt{https://github.com/lgoodfriend/BINS/downloads}\]  After you've downloaded and untarred the tarball, you should see a folder with a bunch of files in it.  The key file for getting started is \texttt{intro.m}.

Start a Matlab session and type \texttt{intro} at the prompt.  You should see four figures appear (Figure~\ref{vel_init}).  These represent the solution to the Taylor-Green vortex after 1 second.  You can change some aspects of how the code runs by changing the inputs to the function \texttt{BINS} in the file \texttt{intro.m}.

\begin{center}
\begin{figure*}
\center
\subfloat[u velocity]{\label{u_init}\includegraphics[width=0.48\textwidth]{intro_u.png} 
}  
\subfloat[v velocity]{\label{v_init}\includegraphics[width=0.48\textwidth]{intro_v.png} 
} \\
\subfloat[speed with streamlines]{\label{u_init}\includegraphics[width=0.48\textwidth]{intro_streamline.png} 
}  
\subfloat[vorticity]{\label{v_init}\includegraphics[width=0.48\textwidth]{intro_vorticity.png} 
} \\
\caption{Results of running \texttt{intro.m}}
\label{vel_init}
\end{figure*}
\end{center}

\subsection{Changing the domain}
The domain size is controlled by \texttt{N} and \texttt{L}.  \texttt{N} sets the number of cells in each direction, and \texttt{L} sets the physical length of each dimension.

\subsection{Changing the boundary conditions}
The boundary conditions are controlled by \texttt{BC}.  \texttt{BC} is a vector with entries corresponding to tags marking the choice of boundary condition:
\[ \texttt{BC} = [\mathrm{lower \hspace{2pt} x, upper\hspace{2pt} x, lower\hspace{2pt} y, upper\hspace{2pt} y} ]\]

BINS has two kinds of boundary conditions
\begin{itemize}
\item periodic
\item moving, impermeable, no-slip wall
\end{itemize}

The periodic boundary conditions are tagged with $\sqrt{-1}$, and the wall boundary conditions are tagged with the velocity of the wall.  Thus, if you wanted an all periodic domain, you would set
\[ \texttt{BC = [sqrt(-1),sqrt(-1),sqrt(-1),sqrt(-1)]}\]
If you wanted the $x$ boundaries to be periodic, with fixed walls on the $y$ boundaries (making an infinite channel), you would set
\[ \texttt{BC = [sqrt(-1),sqrt(-1),0,0]}\]
If you wanted to make a cavity forced by a lid moving at speed $U$ on the upper $y$ boundary, you would set
\[ \texttt{BC = [0,0,0,U]}\]

\subsection{Changing the initial condition}
The initial condition is controlled by \texttt{IC\_choice}, which indexes to an initial condition listed in \texttt{IC.m}.  The available initial conditions are 
\begin{enumerate}
\item $u$ = 0, $v$ = 0, $p$ = 0
\item $u$ = 1, $v$ = 1, $p$ = 0
\item Taylor-Green vortex 
\[ u = \sin(x)\cos(y)\]
\[ v = -\cos(x)\sin(y)\]
\[ p = \frac{1}{4}\left( \cos(2x) + \cos(2y) \right)\]
\item shear flow
\[ u = \begin{cases} \tanh\left(30(y-\frac{1}{4})\right) & \text{if } y \le \frac{L}{2} \\ \tanh\left(30(\frac{3}{4}-y)\right) & \text{otherwise} \end{cases}\]
\[ v = \frac{1}{20}\sin(2\pi x)\]
\[ p = 0\]
\end{enumerate}
You can add a new initial condition by editing \texttt{IC.m} to include the initial condition you would like.  If you are setting a new initial condition, it is important to understand how the staggered grid is set up (see section~\ref{staggeredGrid}).

\subsection{Pre-defined test cases}
BINS includes several test cases already set up:
\begin{enumerate}
\item \texttt{cavity.m}: driven lid cavity flow
\item \texttt{couette.m}: Couette flow
\item \texttt{intro.m}: Taylor-Green vortex
\item \texttt{shear.m}: interesting shear flow
\end{enumerate}

%----------------------------------------------------------------------
\section{Underlying mathematics}
%----------------------------------------------------------------------
\subsection{Notation}
Subscripts are used to denote the location of a variable on the numerical grid.  The variable $u_{ij}$ represents the numerical value of $u$ at point $ij$, where $x=i\Delta x$ and $y=j\Delta y$.

Superscripts are used to denote the numerical timestep.  The variable $u^n$ represents the numerical value of $u$ at timestep $n$, or time $t=n\Delta t$.

\subsection{Solving the Navier-Stokes equation}
\label{mathBasics}

We want to solve the incompressible Navier-Stokes equation
\begin{align*}
  \left. \underbrace{\frac{\partial u}{\partial t}}_\textrm{time} +  \underbrace{\frac{\partial u u}{\partial x} + \frac{\partial u v}{\partial y}}_\textrm{advection} = \underbrace{- \frac{1}{\rho}\frac{\partial p}{\partial x}}_\textrm{pressure} + \underbrace{\nu \left(\frac{\partial^2 u}{\partial x^2} + \frac{\partial^2 u}{\partial y^2}\right) }_\textrm{diffusion} \right\} &\textrm{momentum conservation in x} \\
  \left. \underbrace{\frac{\partial v}{\partial t}}_\textrm{time} +  \underbrace{\frac{\partial v u}{\partial x} + \frac{\partial v v}{\partial y}}_\textrm{advection} = \underbrace{- \frac{1}{\rho}\frac{\partial p}{\partial y}}_\textrm{pressure} + \underbrace{\nu \left(\frac{\partial^2 v}{\partial x^2} + \frac{\partial^2 v}{\partial y^2}\right) }_\textrm{diffusion} \right\} &\textrm{momentum conservation in y} \\
\left. \frac{\partial u}{\partial x} + \frac{\partial v}{\partial y}  = 0 \right\} &\textrm{mass conservation}
\end{align*}

The code first estimates the solution to the Navier-Stokes equation using only the advection and diffusion terms. 
\begin{equation}
 \frac{\partial u^*}{\partial t} =   -\frac{\partial u^n u^n}{\partial x} -\frac{\partial u^n v^n}{\partial y} + \nu\left( \frac{\partial^2 u^n}{\partial x^2} + \frac{\partial^2 u^n}{\partial y^2}  \right) 
 \label{predictx}
\end{equation}
\begin{equation}
  \frac{\partial v^*}{\partial t} =   -\frac{\partial v^n u^n}{\partial x} -\frac{\partial v^n v^n}{\partial y} + \nu\left( \frac{\partial^2 v^n}{\partial x^2} + \frac{\partial^2 v^n}{\partial y^2}  \right)
 \label{predicty}
\end{equation}
The superscript $*$ denotes that the velocities calculated in this step are temporary and don't exist at a timestep.

The code then projects the intermediate velocity field $(u^*,v^*)$ onto a divergence-free space using the next time step pressure $p^{n+1}$ calculated by
\begin{equation}
\frac{1}{\rho}\nabla^2 p^{n+1} = \frac{1}{\Delta t}\left( \frac{\partial u^*}{\partial x} + \frac{\partial v^*}{\partial y} \right)
\label{poisson}
\end{equation}

Then we can find the new divergence free velocity field $(u^{n+1},v^{n+1})$ 

\begin{equation}
u^{n+1}= u^* - \Delta t\frac{1}{\rho}\frac{\partial p^{n+1}}{\partial x}
\label{correctx}
\end{equation}
\begin{equation}
v^{n+1}= v^* - \Delta t\frac{1}{\rho}\frac{\partial p^{n+1}}{\partial y}
\label{correcty}
\end{equation}

since, using equation~\ref{poisson},
\begin{align*}
\frac{\partial  u^{n+1}}{\partial x} +  \frac{\partial  v^{n+1}}{\partial y}&= \frac{\partial}{\partial x}\left(u^* - \Delta t\frac{1}{\rho}\frac{\partial p^{n+1}}{\partial x} \right) + \frac{\partial}{\partial y}\left(v^* - \Delta t\frac{1}{\rho}\frac{\partial p^{n+1}}{\partial y} \right)\\
&= \frac{\partial u^*}{\partial x} + \frac{\partial v^*}{\partial y} - \Delta t\frac{1}{\rho}\nabla^2p^{n+1} \\
&= 0 \\
\end{align*}

These steps appear in the files \texttt{advection.m}, \texttt{diffusionB.m}\footnote{The name \texttt{diffusionB} was chosen to not conflict with Matlab's \texttt{diffusion} function.}, \texttt{predict.m}, \texttt{pressure.m}, and \texttt{correct.m}.  In the $\texttt{predict}$ function, the next time step velocity is estimated following equations~\ref{predictx} and~\ref{predicty}.  This step uses the advection and diffusion terms, calculated in subroutines \texttt{advection} and \texttt{diffusionB} respectively.  The Poisson equation for pressure in equation~\ref{poisson} is solved in the function $\texttt{pressure}$.  The next time step velocity is then calculated by correcting the predicted velocity in function \texttt{correct}, as in equations~\ref{correctx} and~\ref{correcty}.  In the code, the density $\rho$ is assumed to be unit density, i.e. $\rho = 1$, so it does not appear explicitly.  

\renewcommand\arraystretch{2.5}
\begin{center}
\begin{tabular}{|c|c|c|c|}
\hline 
\bf{Component} & \bf{Math} & \bf{Function} & \bf{File}\\ 
\hline 
advection term & \Large{$\frac{\partial u u}{\partial x} + \frac{\partial u v}{\partial y}$}  & \texttt{advection} & \texttt{advection.m}\\ 
 & \Large{$\frac{\partial v u}{\partial x} + \frac{\partial v v}{\partial y}$}  &  & \\ 
\hline 
diffusion term & \Large{$\nu\left( \frac{\partial^2 u}{\partial x^2} + \frac{\partial^2 u}{\partial y^2}\right)$} & \texttt{diffusionB} & \texttt{diffusionB.m} \\ 
 &  \Large{$\nu\left( \frac{\partial^2 v}{\partial x^2} + \frac{\partial^2 v}{\partial y^2}\right)$} &  &  \\
\hline 
next time step: & \Large{$ \frac{\partial u^*}{\partial t} =   -\frac{\partial u^n u^n}{\partial x} -\frac{\partial u^n v^n}{\partial y} + \nu\left( \frac{\partial^2 u^n}{\partial x^2} + \frac{\partial^2 u^n}{\partial y^2} \right)$} & \texttt{predict}& \texttt{predict.m} \\ 
 velocity prediction & \Large{$ \frac{\partial v^*}{\partial t} =   -\frac{\partial v^n u^n}{\partial x} -\frac{\partial v^n v^n}{\partial y} + \nu\left( \frac{\partial^2 v^n}{\partial x^2} + \frac{\partial^2 v^n}{\partial y^2} \right)$} &  &   \\ 
\hline 
next time step: & \Large{$\frac{1}{\rho}\nabla^2 p^{n+1} = \frac{1}{\Delta t}\left( \frac{\partial u^*}{\partial x}+\frac{\partial v^*}{\partial y} \right)$} & \texttt{pressure} & \texttt{pressure.m}\\ 
pressure & & &\\
\hline 
next time step: & \Large{$u^{n+1}= u^* - \Delta t\frac{1}{\rho}\frac{\partial p^{n+1}}{\partial x}$} & \texttt{correct}& \texttt{correct.m} 
  \\ 
  velocity   & \Large{$v^{n+1}= v^* - \Delta t\frac{1}{\rho}\frac{\partial p^{n+1}}{\partial y}$} & & \\
\hline 
\end{tabular} 
\end{center}

\subsection{Finite differences}

To approximate a partial differential equation on the computer, we use finite differences to estimate derivatives.  By default, BINS uses first-order explicit forward Euler time stepping and second-order differences in space.

\subsubsection{Time stepping}
The time stepping in BINS is done using forward Euler time stepping.  If we want to approximate 
\[ \frac{\partial u}{\partial t} = f(u,t) \]
we use the simple approximation
\[ \frac{u^{n+1}-u^n}{\Delta t} = f(u^n,t^n)\]

This allows us to calculate $u$ at the next timestep as
\begin{equation}
u^{n+1} = u^n + \Delta t f(u^n,t^n)
\end{equation}

We use these time stepping routines in the \texttt{predict} function to estimate the next time step velocity using the advection and diffusion terms, without accounting for pressure.  So in BINS we have, for the x-momentum equation,
\[ f(u,v,t) = -\frac{\partial u u}{\partial x}-\frac{\partial v u}{\partial y} + \nu\left( \frac{\partial^2 u}{\partial x^2} + \frac{\partial^2 u}{\partial y^2} \right) \]
and for the y-momentum equation,
\[ f(u,v,t) = -\frac{\partial v u}{\partial x}-\frac{\partial v v}{\partial y} + \nu\left( \frac{\partial^2 v}{\partial x^2} + \frac{\partial^2 v}{\partial y^2} \right) \]

\begin{center}
\begin{tabular}{|c|c|c|c|}
\hline 
\bf{Component} & \bf{Math} & \bf{Function} & \bf{File}\\ 
\hline 
time step & \Large{$\frac{\partial u}{\partial t}$} & \texttt{predict}& \texttt{predict.m} \\ 
 & \Large{$\frac{\partial v}{\partial t}$} &  &   \\ 
\hline 
\end{tabular} 
\end{center}

\subsubsection{Spatial derivatives}
To solve the Navier-Stokes equation, we need to calculate first and second derivatives.  BINS uses centered finite differences for all these calculations.  First derivatives are approximated as
\begin{equation}
\label{firstDeriv}
\frac{\partial u}{\partial x} = \frac{u_{i+1} - u_{i}}{\Delta x}
\end{equation}
and second derivatives as
\begin{equation}
\label{secondDeriv}
\frac{\partial^2 u}{\partial x^2} = \frac{u_{i-1} -2 u_i + u_{i+1}}{\Delta x^2}
\end{equation} 

Subroutines to calculate spatial derivatives are found in the same file as the function that calls them.  BINS has a different function for each first derivative used, because the staggered grid makes the indexing for each derivative slightly different.  This first derivative ends up being second-order because of the staggered grid. For details on how the first derivatives are implemented, see section~\ref{staggeredGrid}.

\begin{center}
\begin{tabular}{|c|c|c|c|}
\hline 
\bf{Component} & \bf{Math} & \bf{Function} & \bf{File}\\ 
\hline 
first derivative: advection term & \Large{$\frac{\partial uu}{\partial x}$} & \texttt{adv\_duudx} & \texttt{advection.m}\\ 
\hline 
first derivative: advection term & \Large{$\frac{\partial uv}{\partial y}$} & \texttt{adv\_duvdy} & \texttt{advection.m}\\ 
\hline 
first derivative: advection term & \Large{$\frac{\partial vu}{\partial x}$} & \texttt{adv\_dvudx} & \texttt{advection.m}\\ 
\hline 
first derivative: advection term & \Large{$\frac{\partial vv}{\partial y}$} & \texttt{adv\_dvvdy} & \texttt{advection.m}\\ 
\hline 
first derivative: pressure correction & \Large{$\frac{\partial p}{\partial x}$} & \texttt{correct\_dpdx} & \texttt{correct.m}\\ 
\hline
first derivative: pressure correction & \Large{$\frac{\partial p}{\partial y}$} & \texttt{correct\_dpdy} & \texttt{correct.m}\\ 
\hline
first derivative: divergence & \Large{$\frac{\partial u}{\partial x} + \frac{\partial v}{\partial y}$} & \texttt{divergence} & \texttt{pressure.m}\\ 
\hline
second derivative & \Large{$\frac{\partial^2 u}{\partial x^2}$} & \texttt{secondDeriv} & \texttt{diffusionB.m}\\ 
\hline 
\end{tabular} 
\end{center}

\subsection{Solving the Poisson equation}
\label{poissonSolve}
\renewcommand\arraystretch{1}
As we saw in section~\ref{mathBasics}, we must solve a Poisson equation to find the pressure at the next timestep (equation~\ref{poisson}):
\[\nabla^2 p^{n+1} = \frac{1}{\Delta t}\left( \frac{\partial u^*}{\partial x} + \frac{\partial v^*}{\partial y}\right)\]

Solving a finite difference Poisson equation is actually inverting a matrix.  To see this, we first write the Poisson equation as a finite difference:
\begin{align*}
\nabla^2 p^{n+1} &= \frac{1}{\Delta t}\left( \frac{\partial u^*}{\partial x} + \frac{\partial v^*}{\partial y}\right) \\
\frac{p_{i+1j}-2p_{ij}+p_{i-1j}}{\Delta x^2} + \frac{p_{ij+1}-2p_{ij}+p_{ij-1}}{\Delta y^2} &= f_{ij}  \\
\end{align*}

We've dropped the superscripts indicating timestep and written the right hand side as $f_{ij}$.  Remember that we know $f_{ij}$ and we need to solve for $p_{ij}$.  Assuming $\Delta x = \Delta y$,
\begin{align*}
\frac{p_{i+1j}-2p_{ij}+p_{i-1j}}{\Delta x^2} + \frac{p_{ij+1}-2p_{ij}+p_{ij-1}}{\Delta x^2} &= f_{ij}  \\
p_{i+1j}-4p_{ij}+p_{i-1j}+ p_{ij+1}+p_{ij-1} &= \Delta x^2 f_{ij}  \\
\end{align*}

By transforming the matrix $p_{ij}$ into a vector, we can convert this into a matrix equation.  For example, consider solving the Poisson equation on a $3 \times 3$ grid:
\begin{center}
\includegraphics[width=0.5\textwidth]{pgrid.png} 
\end{center}

Write the vector p as
\[ p^{T} = [p_{11}, p_{12}, p_{13}, p_{21}, p_{22}, p_{23}, p_{13}, p_{23}, p_{33}]\]

The row for $p_{22}$ is the simplest, since only points inside the domain are needed to calculate second derivatives at the central point.
\[ \left[ \begin{array}{c c c c c c c c c} \\ \\ \\ \\ 0&1&0&1&-4&1&0&1&0 \\ \\ \\ \\ \\ \end{array} \right] \left[ \begin{array}{c} p_{11}\\p_{12}\\p_{13}\\p_{21}\\p_{22}\\p_{23}\\p_{31}\\p_{32}\\p_{33}\\ \end{array}\right] = \left[ \begin{array}{c} f_{11}\\f_{12}\\f_{13}\\f_{21}\\f_{22}\\f_{23}\\f_{31}\\f_{32}\\f_{33}\\ \end{array}\right]
\]
For all the other points, we know some information, but other information must be filled in using boundary conditions.  If we just filled in the matrix using the finite difference formula as much as possible, we would find
\[ \left[ \begin{array}{c c c c c c c c c} 
-4&1&0&1&0&0&0&0&0\\ 
1&-4&1&0&1&0&0&0&0\\ 
0&1&-4&0&0&1&0&0&0\\ 
1&0&0&-4&1&0&1&0&0\\ 
0&1&0&1&-4&1&0&1&0 \\ 
0&0&1&0&1&-4&0&1&0\\ 
0&0&0&1&0&0&-4&1&0\\ 
0&0&0&0&1&0&1&-4&1\\ 
0&0&0&0&0&1&0&1&-4\\ \end{array} \right] \left[ \begin{array}{c} p_{11}\\p_{12}\\p_{13}\\p_{21}\\p_{22}\\p_{23}\\p_{31}\\p_{32}\\p_{33}\\ \end{array}\right] = \left[ \begin{array}{c} f_{11}\\f_{12}\\f_{13}\\f_{21}\\f_{22}\\f_{23}\\f_{31}\\f_{32}\\f_{33}\\ \end{array}\right]
\]

This matrix equation is incorrect because it does not enforce boundary conditions\footnote{In fact, it does not form a true finite difference approximation to derivatives, since the rows do not all sum to 0.}. For all but the center point in this simple example, we need boundary conditions to calculate the Laplacian.  

For example, if the boundary conditions are periodic (see section~\ref{periodicBCs}), then $p_{12}$ is adjacent to $p_{32}$, $p_{21}$ is adjacent to $p_{23}$, etc.  Then the matrix is
\[ \left[ \begin{array}{c c c c c c c c c} 
-4&1&1&1&0&0&1&0&0\\ 
1&-4&1&0&1&0&0&1&0\\ 
1&1&-4&0&0&1&0&0&1\\ 
1&0&0&-4&1&1&1&0&0\\ 
0&1&0&1&-4&1&0&1&0 \\ 
0&0&1&1&1&-4&0&1&0\\ 
1&0&0&1&0&0&-4&1&1\\ 
0&1&0&0&1&0&1&-4&1\\ 
0&0&1&0&0&1&1&1&-4\\ \end{array} \right] \left[ \begin{array}{c} p_{11}\\p_{12}\\p_{13}\\p_{21}\\p_{22}\\p_{23}\\p_{31}\\p_{32}\\p_{33}\\ \end{array}\right] = \left[ \begin{array}{c} f_{11}\\f_{12}\\f_{13}\\f_{21}\\f_{22}\\f_{23}\\f_{31}\\f_{32}\\f_{33}\\ \end{array}\right]
\]
We can then solve for the pressure by solving this matrix equation
\[ Ap = f\]
which has solution
\[ p = A^{-1}f\]
In BINS, the matrix $A$ and vector $f$ are formed in \texttt{make\_matrix.m}, and the solution is found in \texttt{solvePoisson.m}
\begin{center}
\begin{tabular}{|c|c|c|c|}
\hline 
\bf{Component} & \bf{Math} & \bf{Function} & \bf{File}\\ 
\hline 
matrix defining Laplacian & $A$ & \texttt{make\_matrix} & \texttt{make\_matrix.m}\\ 
\hline 
vector defining Poisson forcing & $f$ & \texttt{make\_matrix} & \texttt{make\_matrix.m}\\ 
\hline 
solution to Poisson equation & $p = A^{-1}f$ & \texttt{solvePoisson} & \texttt{solvePoisson.m}\\ 
\hline 
\end{tabular} 
\end{center}

\subsubsection{Inverting a matrix}
Solving the Poisson equation requires inverting a matrix.  Gaussian elimination can invert a matrix, but this algorithm is expensive, in that it takes a lot of computational time.  Many shortcuts have been developed to invert matrices quickly.  These shortcuts often do not solve the inversion exactly, which is ok, because we are only approximating the solution to the Navier-Stokes equations anyway.

This version of BINS uses Matlab's built in matrix inversion function \texttt{backslash}, also written ``\textbackslash''.  This is admittedly a crutch, but it's a very nice crutch.  Matlab has sparse matrix functions built in, which makes inverting sparse matrices like $A$ very fast.  Solving the Poisson equation is the slowest part of BINS and many other Navier-Stokes solvers.

%----------------------------------------------------------------------
\section{Specifics of the computation}
%----------------------------------------------------------------------
\FloatBarrier
\subsection{Staggered grid}
\label{staggeredGrid}
In a standard numerical grid, all the variables are located at cell centers (Figure~\ref{collocated}).  This arrangement is simple to code, but it is not commonly used in Navier-Stokes solvers.  A staggered grid (Figure~\ref{staggered}) is more natural for solving the Navier-Stokes equation, because variables are located such that compact, high-order schemes can be used.

\begin{center}
\begin{figure*}[h]
\center
\subfloat[Collocated grid]{\label{collocated}\includegraphics[width=0.45\textwidth]{colocated.png} 
}  
\subfloat[Staggered grid]{\label{staggered}\includegraphics[width=0.45\textwidth]{staggered.png} 
} 
\caption{Different types of variable centering on a grid.}
\label{grid_types}
\end{figure*}
\end{center}

\FloatBarrier
The staggered grid means that each of the four derivatives needed in the advective terms are slightly different.  The $x$-direction advective term is defined on the left side of cells, at the same point as $u$ (Figure~\ref{staggered}).  To calculate the $\frac{\partial uu}{\partial x}$ term, we can first average $u$ to cell centers, and the take the derivative at the left cell face (Figure~\ref{duudx}).  Thus we calculate
\[ u_{\text{minus}} = \frac{1}{2}\left(u_{i-1,j}+u_{i,j} \right) \]
\[ u_{\text{plus}} = \frac{1}{2}\left(u_{i+1,j}+u_{i,j} \right) \]
\[ \left.\frac{\partial uu}{\partial x}\right|_{i,j} = \frac{1}{\Delta x}\left( u_{\text{plus}}u_{\text{plus}} - u_{\text{minus}}u_{\text{minus}}\right)\]
Similarly, to calculate $\frac{\partial uv}{\partial y}$, we average $u$ and $v$ to cell corners, and then take the derivative at the left cell face (Figure~\ref{duvdy}).  The computation is

\[ u_{\text{minus}} = \frac{1}{2}\left(u_{i,j-1}+u_{i,j} \right) \]
\[ u_{\text{plus}} = \frac{1}{2}\left(u_{i,j+1}+u_{i,j} \right) \]
\[ v_{\text{minus}} = \frac{1}{2}\left(v_{i-1,j}+v_{i,j} \right) \]
\[ v_{\text{plus}} = \frac{1}{2}\left(v_{i-1,j+1}+v_{i,j+1} \right) \]
\[ \left.\frac{\partial uv}{\partial y}\right|_{i,j} = \frac{1}{\Delta y}\left( u_{\text{plus}}v_{\text{plus}} - u_{\text{minus}}v_{\text{minus}}\right)\]

A similar averaging procedure is used for the $y$-direction advection terms (Figure~\ref{y_adv}).  Note that all of these derivatives are centered between the plus and minus velocities, making the derivatives second order.


\begin{center}
\begin{figure*}[h]
\center
\subfloat[$\frac{\partial uu}{\partial x}$]{\label{duudx}\includegraphics[width=0.45\textwidth]{duudx.png} 
}  
\subfloat[$\frac{\partial uv}{\partial y}$]{\label{duvdy}\includegraphics[width=0.45\textwidth]{duvdy.png} 
} 
\caption{First derivatives for $x$ direction advection term.}
\label{x_adv}
\end{figure*}
\end{center}

\begin{center}
\begin{figure*}
\center
\subfloat[$\frac{\partial vu}{\partial x}$]{\label{dvudx}\includegraphics[width=0.45\textwidth]{dvudx.png} 
}  
\subfloat[$\frac{\partial vv}{\partial y}$]{\label{dvvdy}\includegraphics[width=0.45\textwidth]{dvvdy.png} 
} 
\caption{First derivatives for $x$ direction advection term.}
\label{y_adv}
\end{figure*}
\end{center}

The derivatives needed for the divergence-free / pressure solve are relatively simple.  The divergence is needed at cell centers, which is naturally second-order with face-centered velocities (Figure~\ref{div}). We can compute

\[ \text{divergence}_{i,j} = \frac{1}{\Delta x}\left( u_{i+1,j}-u_{i,j}\right) + \frac{1}{\Delta y}\left( v_{i,j+1}-v_{i,j}\right) \]

Similarly, the pressure derivatives needed at cell faces are naturally second-order with cell-centered pressure (Figure~\ref{dpdx}), computing
\[ \left.\frac{\partial p}{\partial x}\right|_{i,j} = \frac{1}{\Delta x}\left( p_{i,j} - p_{i-1,j}\right)\]
\[ \left.\frac{\partial p}{\partial y}\right|_{i,j} = \frac{1}{\Delta y}\left( p_{i,j} - p_{i,j-1}\right)\]

\begin{center}
\begin{figure*}
\center
\subfloat[divergence]{\label{div}\includegraphics[width=0.45\textwidth]{div.png} 
}  
\subfloat[pressure derivatives]{\label{dpdx}\includegraphics[width=0.45\textwidth]{dpdx.png} 
} 
\caption{First derivatives for the divergence-free / pressure solve.}
\label{pressDeriv}
\end{figure*}
\end{center}



\FloatBarrier
\subsection{Boundary conditions}
\label{BCs}
When we want to calculate a first derivative of $u_i$ ($u$ at grid point $i$), we require $u_{i+1}$.  If $u_i$ is on an upper boundary, we need to know what to use for $u_{i+1}$ in equation~\ref{firstDeriv}.  The same is true for calculating a second derivative in equation~\ref{secondDeriv}.

BINS uses two kinds of boundary conditions: periodic and moving wall.  Boundary conditions are implemented using ``ghost cells'' (Figure~\ref{ghostcells}).  Ghost cells are points which lie outside the real solution domain, but are needed to calculate derivatives at the boundaries.  The solution is calculated on the inner domain, and then ghost cells are filled so that derivatives can be taken in the next time step.  Boundary conditions are enforced in \texttt{BCfill.m}.

\begin{center}
\begin{figure*}
\center
\includegraphics[width=0.48\textwidth]{ghostcell.png} 
\caption{Ghostcells outside the boundary of the domain.}
\label{ghostcells}
\end{figure*}
\end{center}

\subsubsection{Periodic boundary condition}
\label{periodicBCs}
A periodic boundary condition states that opposite edges of the domain are adjacent (Figure~\ref{torus}).  It is called ``periodic'' because this boundary condition causes the solution to repeat infinitely, so it becomes periodic with a period equal to the length of the domain.
\begin{center}
\begin{figure*}
\center
\includegraphics[width=0.48\textwidth]{torus.png} 
\caption{Schematic of periodic domain.}
\label{torus}
\end{figure*}
\end{center}

To implement this boundary condition, the solution at the right edge of the domain is copied to the ghost cells on the left edge of the domain, and vice-versa (Figure~\ref{pBC}).
\begin{center}
\begin{figure*}
\center
\includegraphics[width=0.48\textwidth]{periodic_BCs.png} 
\caption{How to fill periodic BCs.}
\label{pBC}
\end{figure*}
\end{center}  

\subsubsection{Moving wall boundary condition}
A moving wall boundary condition states that the boundary is an impermeable wall that is moving parallel to itself (Figure~\ref{movingwall}).  It can also be a stationary wall by setting the wall velocity to zero.
\begin{center}
\begin{figure*}
\center
\includegraphics[width=0.48\textwidth]{movingwall.png} 
\caption{Schematic of moving wall domain.}
\label{movingwall}
\end{figure*}
\end{center}


Unlike for the periodic boundary condition, the moving wall boundary condition must treat the two velocities and pressure differently (Figure~\ref{movingwall_loc}).  The velocity perpendicular to the wall must be zero at the wall, otherwise fluid would be moving into the wall.  The velocity parallel to the wall must be equal to the wall's velocity at the wall to satisfy the no-slip condition.  The derivative of the pressure perpendicular to the wall must be zero at the wall, otherwise the pressure would generate a force moving fluid into the wall.
\begin{center}
\begin{figure*}
\center
\includegraphics[width=0.48\textwidth]{movingwall_loc.png} 
\caption{Schematic of moving wall boundary condition locations.}
\label{movingwall_loc}
\end{figure*}
\end{center}

\begin{center}
\begin{figure*}
\center
\includegraphics[width=0.48\textwidth]{movingwall_BC.png} 
\caption{Schematic of moving wall boundary condition interpolation.}
\label{movingwall_BC}
\end{figure*}
\end{center}

The way we structured the grid means that we can set the velocity perpendicular to the wall to be equal to zero, because that velocity has a grid point on the wall (Figure~\ref{movingwall_BC}).  The velocity parallel to the wall does not have a grid point on the wall, so we need to interpolate. We have chosen to use a linear interpolation, so we require
\[ \frac{u_1 + u_{\mathrm{ghost}}}{2} = U_{\mathrm{wall}}\]
which means that
\begin{equation}
u_{\mathrm{ghost}} = 2U_{\mathrm{wall}} - u_1
\end{equation}
%To set the pressure, we use the approximation to the first derivative in equation~\ref{firstDeriv}, and require
%\[ \frac{p_1 - p_{\mathrm{ghost}}}{dx} = 0\]
%which means that
%\begin{equation}
%p_{\mathrm{ghost}} = p_1
%\end{equation}
Because we set the velocity perpendicular to the wall to be equal to zero on the wall, we do not need the pressure in the ghost cells.  That pressure would only have been used for correcting the velocity perpendicular to the wall, but the resulting value is overwritten by the boundary condition.

%----------------------------------------------------------------------
\section{Known limitations}
%----------------------------------------------------------------------
\begin{itemize}
\item Code assumes $\Delta x=\Delta y$
\item Code assumes $T = k\Delta t$ for integer $k$
\item Code requires $L_x = L_y$ and $N_x = N_y$ (square domains, isotropic cells)
\item Because code is 1st order, it is dissipative.
\item Code is slow, partially due to reliance on Matlab's ``\textbackslash''.
\item Code does not run in Octave, due to differences in how Octave and Matlab find functions.
\end{itemize}

%----------------------------------------------------------------------
\section{Validation}
%----------------------------------------------------------------------
We have validated BINS using the Taylor-Green vortex.  This solution is excellent for validating 2D, incompressible Navier-Stokes solvers because it is time-varying , with a fairly complex structure, but still has an exact solution.  
The exact solution on a periodic domain of dimension $[0,2\pi] \times [0,2\pi]$ is
\[ u=\sin (x) \cos (y) F(t)\]
\[ v=-\cos (x) \sin (y) F(t)\]
\[ F(t)=e^{-2\nu t}\]

\subsection{Comparison with exact solution}
For a first check, we can look at the difference between the numerical and exact solution using \texttt{compTest.m} (Figure~\ref{error}).  There is little error between the exact and numerical solutions.

\begin{center}
\begin{figure*}
\center
\includegraphics[width=0.6\textwidth]{compTest.png} 
\caption{Numerical and exact solution at $t=2$ on a grid of size $N=64$.  Solutions are at $y=\pi/2$, varying in $x$.}
\label{error}
\end{figure*}
\end{center}

\subsection{Convergence test}
A common test is to observe how the solution accuracy increases with decreasing grid spacing.  The rate at which the error decreases is called the convergence rate, also known as the order of accuracy of the solution algorithm.  

Difference schemes have different convergence rates, which can be calculated theoretically.  For example, the forward Euler timestepping used in BINS is first-order, while the spatial differences are second order\footnote{The first derivatives are second order accurate because of the staggered grid.}.  But what is the order of accuracy of BINS?

We can measure the order of accuracy of our real code using a convergence test in \texttt{convTest.m}.  We solve the same test problem, the Taylor-Green vortex, up to time $t=2$ for a range of spatial and time steps.  We calculate the error of our solution, using the known exact solution.  We can now plot the error as a function of the grid spacing/time step (Figure~\ref{conv}).  The order of accuracy of our code is the slope of the best fit line through this plot.  This plot shows the order to be 1.02, which is close to the order 1 predicted by using forward Euler time stepping.

\begin{figure}
\center
\includegraphics[width=.7\textwidth]{convTest.png} 
\caption{Convergence test for BINS, order = 1.02.}
\label{conv}
\end{figure}


\FloatBarrier
%----------------------------------------------------------------------
\section{BINS reference: by function name}
%----------------------------------------------------------------------
This section provides a function-by-function breakdown of the components that make BINS work.  You can see a map of how the functions call each other in a call graph (Figure~\ref{callGraph}).

\begin{figure}
\center
\includegraphics[width=\textwidth]{callGraph.png} 
\caption{Call graph for BINS.}
\label{callGraph}
\end{figure}

\subsection{\texttt{adv\_duudx}}
\begin{description}
\item[In file:] \hfill \\ \texttt{advection.m}
\item[Called by:] \hfill \\ \texttt{advection}
\item[Inputs:] \hfill \\ $u$: $x$-direction velocity \\ $h$: spatial step $\Delta x = \Delta y$
\item[Outputs:] \hfill \\ $duudx$: advection term $\frac{\partial uu}{\partial x}$
\item[Calls:] \hfill \\ nothing
\end{description}
This function calculates one first derivative required for the advection term of the x-momentum equation.

\subsection{\texttt{adv\_duvdy}}
\begin{description}
\item[In file:] \hfill \\ \texttt{advection.m}
\item[Called by:] \hfill \\ \texttt{advection}
\item[Inputs:] \hfill \\ $u$: $x$-direction velocity \\ $v$: $y$-direction velocity \\$h$: spatial step $\Delta x = \Delta y$
\item[Outputs:] \hfill \\ $duvdy$: advection term $\frac{\partial uv}{\partial y}$
\item[Calls:] \hfill \\ nothing
\end{description}
This function calculates one first derivative required for the advection term of the x-momentum equation.

\subsection{\texttt{adv\_dvudx}}
\begin{description}
\item[In file:] \hfill \\ \texttt{advection.m}
\item[Called by:] \hfill \\ \texttt{advection}
\item[Inputs:] \hfill \\ $u$: $x$-direction velocity \\ $v$: $y$-direction velocity \\$h$: spatial step $\Delta x = \Delta y$
\item[Outputs:] \hfill \\ $dvudx$: advection term $\frac{\partial vu}{\partial x}$
\item[Calls:] \hfill \\ nothing
\end{description}
This function calculates one first derivative required for the advection term of the y-momentum equation.

\subsection{\texttt{adv\_dvvdy}}
\begin{description}
\item[In file:] \hfill \\ \texttt{advection.m}
\item[Called by:] \hfill \\ \texttt{advection}
\item[Inputs:] \hfill \\ $v$: $y$-direction velocity \\ $h$: spatial step $\Delta x = \Delta y$
\item[Outputs:] \hfill \\ $dvvdy$: advection term $\frac{\partial vv}{\partial y}$
\item[Calls:] \hfill \\ nothing
\end{description}
This function calculates one first derivative required for the advection term of the y-momentum equation.

\subsection{\texttt{advection}}
\begin{description}
\item[In file:] \hfill \\ \texttt{advection.m}
\item[Called by:] \hfill \\ \texttt{BINS}
\item[Inputs:] \hfill \\ $u$: $x$-direction velocity \\ $v$: $y$-direction velocity \\ $h$: spatial step $\Delta x = \Delta y$ 
\item[Outputs:] \hfill \\ $u\_advection$: $x$-direction advection term \\ $v\_advection$: $y$-direction advection term 
\item[Calls:] \hfill \\ \texttt{adv\_duudx} \\ \texttt{adv\_duvdy} \\ \texttt{adv\_dvudx} \\ \texttt{adv\_dvvdy}
\end{description}

This function  estimates the advection term of the x- and y-momentum equations.

\[ \text{advection term for x-momentum equation} = \frac{\partial uu}{\partial x} + \frac{\partial vu}{\partial y} \]
\[ \text{advection term for y-momentum equation} = \frac{\partial uv}{\partial x} + \frac{\partial vv}{\partial y} \]


\subsection{\texttt{BINS}}
\begin{description}
\item[In file:] \hfill \\ \texttt{BINS.m}
\item[Called by:] \hfill \\ none
\item[Inputs:] \hfill \\ $N$: number of cells in $x$ and $y$ \\ $L$: physical dimension in $x$ and $y$ \\ $dt$: time step \\ $T$: final time \\ $BC$: vector of boundary conditions \\ $IC\_choice$: flag choosing predefined IC \\ $nu$: molecular kinematic viscosity \\ $ng$: number of ghost cells
\item[Outputs:] \hfill \\ none
\item[Calls:] \hfill \\ \texttt{IC} \\ \texttt{make\_matrix} \\ \texttt{fillBC} \\ \texttt{advection} \\ \texttt{diffusionB} \\ \texttt{predict} \\ \texttt{pressure} \\ \texttt{correct} \\ \texttt{plotSoln}
\end{description}
This is the main function.  It calls the functions needed to advance the solution in time, and plots the solution at the last timestep.

\subsection{\texttt{correct}}
\begin{description}
\item[In file:] \hfill \\ \texttt{correct.m}
\item[Called by:] \hfill \\ \texttt{BINS}
\item[Inputs:] \hfill \\ $uStar$: predicted $x$-direction velocity \\ $vStar$: predicted $y$-direction velocity \\ $newP$: pressure at next time step \\ $h$: spatial step $\Delta x = \Delta y$ \\ $dt$: time step
\item[Outputs:] \hfill \\ $uNext$: $x$-direction velocity at next time step \\ $vNext$: $y$-direction velocity at next time step
\item[Calls:] \hfill \\ \texttt{correct\_dpdx} \\ \texttt{correct\_dpdy}
\end{description}
This function corrects the next step velocity estimated in \texttt{predict} using the next step pressure calculated in \texttt{pressure} using equations~\ref{correctx} and~\ref{correcty}:

\[ u^{n+1}= u^* - \Delta t\frac{\partial p^{n+1}}{\partial x} \]
\[ v^{n+1}= v^* - \Delta t\frac{\partial p^{n+1}}{\partial y} \]

\subsection{\texttt{correct\_dpdx}}
\begin{description}
\item[In file:] \hfill \\ \texttt{correct.m}
\item[Called by:] \hfill \\ \texttt{correct}
\item[Inputs:] \hfill \\ $p$: pressure \\ $h$: spatial step $\Delta x = \Delta y$
\item[Outputs:] \hfill \\ $dpdx$: $\frac{\partial p}{\partial x}$
\item[Calls:] \hfill \\ none
\end{description}
This function calculates the pressure derivative
\[ \frac{\partial p}{\partial x}\]
needed in the correction step in function \texttt{correct}.

\subsection{\texttt{correct\_dpdy}}
\begin{description}
\item[In file:] \hfill \\ \texttt{correct.m}
\item[Called by:] \hfill \\ \texttt{correct}
\item[Inputs:] \hfill \\ $p$: pressure \\ $h$: spatial step $\Delta x = \Delta y$
\item[Outputs:] \hfill \\ $dpdy$: $\frac{\partial p}{\partial y}$
\item[Calls:] \hfill \\ none
\end{description}
This function calculates the pressure derivative
\[ \frac{\partial p}{\partial y}\]
needed in the correction step in function \texttt{correct}.

\subsection{\texttt{diffusionB}}
\begin{description}
\item[In file:] \hfill \\ \texttt{diffusionB.m}
\item[Called by:] \hfill \\ \texttt{BINS}
\item[Inputs:] \hfill \\ $u$: $x$-direction velocity \\ $v$: $y$-direction velocity \\ $h$: spatial step $\Delta x = \Delta y$ \\ $nu$: molecular kinematic viscosity
\item[Outputs:] \hfill \\ $u\_diffusion$: $x$-direction diffusion term \\ $v\_diffusion$: $y$-direction diffusion term 
\item[Calls:] \hfill \\ \texttt{secondDeriv}
\end{description}
This function calculates the diffusion term of the x- and y-momentum equations.
\[ \text{diffusion term for x-momentum equation} = \nu \left(\frac{\partial^2 u}{\partial x^2} + \frac{\partial^2 u}{\partial y^2}\right) \]
\[ \text{diffusion term for y-momentum equation} = \nu \left(\frac{\partial^2 v}{\partial x^2} + \frac{\partial^2 v}{\partial y^2}\right) \]
\subsection{\texttt{divergence}}
\begin{description}
\item[In file:] \hfill \\ \texttt{pressure.m}
\item[Called by:] \hfill \\ \texttt{pressure}
\item[Inputs:] \hfill \\ $u$: $x$-direction velocity \\ $v$: $y$-direction velocity \\ $h$: spatial step $\Delta x = \Delta y$ 
\item[Outputs:] \hfill \\ $divOut$: divergence of input velocity field $(u,v)$
\item[Calls:] \hfill \\ none
\end{description}
This function calculates the divergence of the intermediate, predicted velocity field $(u^*,v^*)$
\[ \text{divergence} = \frac{\partial u^*}{\partial x} + \frac{\partial v^*}{\partial y}\]
This is needed to calculate the next time step pressure in the function \texttt{pressure}.

\subsection{\texttt{fillBC}}
\begin{description}
\item[In file:] \hfill \\ \texttt{fillBC.m}
\item[Called by:] \hfill \\ \texttt{BINS}
\item[Inputs:] \hfill \\ $u$: $x$-direction velocity \\ $v$: $y$-direction velocity \\ $p$: pressure \\ $h$: spatial step $\Delta x = \Delta y$ \\ $ng$: number of ghost cells \\ $N$: number of cells in $x$ and $y$ \\ $BC$: vector defining boundary conditions
\item[Outputs:] \hfill \\ $uOut$: input $u$ with filled ghost cells \\ $vOut$: input $v$ with filled ghost cells \\ $pOut$: input $p$ with filled ghost cells
\item[Calls:] \hfill \\ none
\end{description}
This function fills the ghost cells of the variables $u$, $v$, and $p$, while enforcing the boundary conditions.  For more details on the boundary conditions, see section~\ref{BCs}.

\subsection{\texttt{IC}}
\begin{description}
\item[In file:] \hfill \\ \texttt{IC.m}
\item[Called by:] \hfill \\ \texttt{BINS}
\item[Inputs:] \hfill \\ $L$: length of domain in $x$ and $y$ \\ $N$: number of grid cells in $x$ and $y$ \\ $ng$: number of ghost cells \\ $IC\_choice$: flag defining choice of predefined initial condition
\item[Outputs:] \hfill \\ $u$: $x$-direction velocity \\ $v$: $y$-direction velocity \\ $p$: pressure
\item[Calls:] \hfill \\ none
\end{description}
This function is in the file \texttt{IC.m}.  It initializes the $u$, $v$, and $p$ variables with pre-set initial condition.  The default options are
\begin{enumerate}
\item $u$ = 0, $v$ = 0, $p$ = 0
\item $u$ = 1, $v$ = 1, $p$ = 0
\item Taylor-Green vortex 
\[ u = \sin(x)\cos(y)\]
\[ v = -\cos(x)\sin(y)\]
\[ p = \frac{1}{4}\left( \cos(2x) + \cos(2y) \right)\]
\item shear flow
\[ u = \begin{cases} \tanh\left(30(y-\frac{1}{4})\right) & \text{if } y \le \frac{L}{2} \\ \tanh\left(30(\frac{3}{4}-y)\right) & \text{otherwise} \end{cases}\]
\[ v = \frac{1}{20}\sin(2\pi x)\]
\[ p = 0\]
\end{enumerate}

\subsection{\texttt{make\_matrix}}
\begin{description}
\item[In file:] \hfill \\ \texttt{make\_matrix.m}
\item[Called by:] \hfill \\ \texttt{BINS}
\item[Inputs:] \hfill \\ $bigM$: size of matrix $= N^2$ \\ $N$: number of grid points in $x$ and $y$ \\ $ng$: number of ghost cells \\ $h$: grid spacing $\Delta x = \Delta y$ \\ $Mlookup$: table for finding $(i,j)$ coordinate of point in matrix $A$ or vector $f$ \\ $m$: table for finding location of point in matrix $A$ or vector $f$ from $(i,j)$ coordinate \\ $rhs$: right hand side of Poisson equation \\ $BC$: vector of boundary conditions
\item[Outputs:] \hfill \\ $A$: matrix of Poisson equation \\ $f$: vector of Poisson equation
\item[Calls:] \hfill \\ none
\end{description}
This function generates the $A$ matrix and $f$ vector that define the matrix equation
\[ Ap^{n+1} = f\]
that encodes the Poisson equation
\[ \nabla^2 p^{n+1} = \frac{1}{\Delta t}\left( \frac{\partial u^*}{\partial x} + \frac{\partial v^*}{\partial y} \right) \]
For details, see section~\ref{poissonSolve}.

\subsection{\texttt{plotSoln}}
\begin{description}
\item[In file:] \hfill \\ \texttt{plotSoln.m}
\item[Called by:] \hfill \\ \texttt{BINS}
\item[Inputs:] \hfill \\  $u$: $x$-direction velocity \\ $v$: $y$-direction velocity\\ $ng$: number of ghost cells \\ $L$: size of domain in $x$ and $y$ \\ $N$: number of cells in $x$ and $y$ \\ $h$: spatial step $\Delta x = \Delta y$ 
\item[Outputs:] \hfill \\ none
\item[Calls:] \hfill \\ none
\end{description}
This function plots the velocities $u$ and $v$ at end time $T$.

\subsection{\texttt{poissonSolve}}
\begin{description}
\item[In file:] \hfill \\ \texttt{poissonSolve.m}
\item[Called by:] \hfill \\ \texttt{pressure}
\item[Inputs:] \hfill \\  $rhs$: right hand side to Poisson equation \\ $h$: spatial step $\Delta x = \Delta y$ \\ $BC$: vector defining boundary conditions \\ $ng$: number of ghost cells \\ $N$: number of cells in $x$ and $y$ \\ $A$: matrix defining Poisson operator
\item[Outputs:] \hfill \\ $solution$: solution to Poisson equation
\item[Calls:] \hfill \\ \texttt{Matlab \textbackslash}
\end{description}
This function solves the Poisson equation
\[ \nabla^2 p^{n+1} = \frac{1}{\Delta t}\left( \frac{\partial u^*}{\partial x} + \frac{\partial v^*}{\partial y} \right) \]
by encoding it as a matrix inversion
\[ Ap^{n+1} = f\]
and then solving
\[ p^{n+1} = A^{-1}f\]

\subsection{\texttt{predict}}
\begin{description}
\item[In file:] \hfill \\ \texttt{predict.m}
\item[Called by:] \hfill \\ \texttt{BINS}
\item[Inputs:] \hfill \\ $u$: $x$-direction velocity\\$v$: $y$-direction velocity\\ $u\_advection$: $x$-direction advection term \\ $v\_advection$: $y$-direction advection term \\$u\_diffusion$: $x$-direction diffusion term \\$v\_diffusion$: $y$-direction diffusion term \\$dt$: time step
\item[Outputs:] \hfill \\ $uStar$: intermediate $x$-direction velocity \\ $vStar$: intermediate $y$-direction velocity
\item[Calls:] \hfill \\ none
\end{description}
This function predicts the next time step velocity using equations~\ref{predictx} and~\ref{predicty}:
\[ 
 \frac{\partial u^*}{\partial t} =   -\frac{\partial u^n u^n}{\partial x} -\frac{\partial u^n v^n}{\partial y} + \nu\left( \frac{\partial^2 u^n}{\partial x^2} + \frac{\partial^2 u^n}{\partial y^2}  \right)  \]

\[
  \frac{\partial v^*}{\partial t} =   -\frac{\partial v^n u^n}{\partial x} -\frac{\partial v^n v^n}{\partial y} + \nu\left( \frac{\partial^2 v^n}{\partial x^2} + \frac{\partial^2 v^n}{\partial y^2}  \right)
\]
This uses only the advection and diffusion terms, so the pressure gradient is not included.  At the end of this step, the intermediate velocity field $(u^*,v^*)$ is not divergence-free, i.e. mass is not conserved.

\subsection{\texttt{pressure}}
\begin{description}
\item[In file:] \hfill \\ \texttt{pressure.m}
\item[Called by:] \hfill \\ \texttt{BINS}
\item[Inputs:] \hfill \\ $uStar$: intermediate $x$-direction velocity \\ $vStar$: intermediate $y$-direction velocity \\ $p$: pressures \\ $h$: spatial step $\Delta x = \Delta y$ \\ $dt$: time step \\ $ng$: number of ghost cells \\ $BC$: vector defining boundary conditions \\ $N$: number of cells in $x$ and $y$ \\ $A$: matrix defining Poisson operator
\item[Outputs:] \hfill \\ $newP$: pressure at next time step
\item[Calls:] \hfill \\ \texttt{divergence} \\ \texttt{poissonSolve}
\end{description}
This function solves for the pressure at the next time step.

\subsection{\texttt{secondDeriv}}
\begin{description}
\item[In file:] \hfill \\ \texttt{diffusionB.m}
\item[Called by:] \hfill \\ \texttt{diffusionB}
\item[Inputs:] \hfill \\ $u$: velocity (could be $x$- or $y$-direction) \\ $v$: $y$-direction velocity \\ $dir$: direction of derivative  \\ $h$: spatial step $\Delta x = \Delta y$ 
\item[Outputs:] \hfill \\ $divOut$: second derivative of input velocity field
\item[Calls:] \hfill \\ none
\end{description}
This function calculates the unmixed second derivatives of the velocity field for use in the diffusion term.

%----------------------------------------------------------------------
\section{BINS reference: by filename}
%----------------------------------------------------------------------
This section provides a file-by-file breakdown of the components that make BINS work.

\subsection{\texttt{advection.m}}
\begin{description}
\item[Functions:] \hfill \\ \texttt{advection} \\ \texttt{adv\_duudx} \\ \texttt{adv\_duvdy} \\  \texttt{adv\_dvudx} \\  \texttt{adv\_dvvdy}
\end{description}
This file contains the functions needed to calculate the advection term.

\subsection{\texttt{BINS.m}}
This file contains the function \texttt{BINS}, the main function of BINS.

\subsection{\texttt{cavity.m}}
This file contains a script that solves a driven lid cavity problem, which is intended to demonstrate the solid wall boundary conditions.

\subsection{\texttt{compTest.m}}
This file contains a script that compares the Taylor-Green vortex solution found in \texttt{BINS} to its analytical solution.

\subsection{\texttt{convTest.m}}
This file contains a script that performs a convergence test of \texttt{BINS}.  It calculates the error for each grid spacing by comparing the Taylor-Green vortex solution found in \texttt{BINS} to its analytical solution.

\subsection{\texttt{correct.m}}
\begin{description}
\item[Functions:] \hfill \\ \texttt{correct} \\ \texttt{correct\_dpdx} \\ \texttt{correct\_dpdy} 
\end{description}
This file contains the functions needed to correct the next time step velocity using the updated pressure.

\subsection{\texttt{couette.m}}
This file contains a script that solves a Couette flow problem, which is intended to demonstrate the solid wall boundary conditions.

\subsection{\texttt{diffusionB.m}}
\begin{description}
\item[Functions:] \hfill \\ \texttt{diffusionB} \\ \texttt{secondDeriv} 
\end{description}
This file contains the functions needed to calculate the diffusion term.

\subsection{\texttt{fillBC.m}}
This file contains the function \texttt{fillBC}, which fills the ghostcells of $u$, $v$, and $p$ to enforce boundary conditions.

\subsection{\texttt{IC.m}}
This file contains the function \texttt{IC}, which initializes $u$, $v$, and $p$ with predefined values.

\subsection{\texttt{intro.m}}
This file contains a script that solves the Taylor-Green vortex.  It is intended to introduce how \texttt{BINS} is called.

\subsection{\texttt{make\_matrix.m}}
This file contains the function \texttt{make\_matrix}, which generates the matrix inversion problem used to solve the Poisson pressure equation.

\subsection{\texttt{multigrid.m}}
This file is not currently in use.

\subsection{\texttt{plotSoln.m}}
This file contains the function \texttt{plotSoln}, which plots $u$ and $v$ at the final time $T$.

\subsection{\texttt{poissonSolve.m}}
This file contains the function \texttt{poissonSolve}, which solves the Poisson pressure equation.

\subsection{\texttt{predict.m}}
This file contains the function \texttt{predict}, which predicts the next time step velocity field based on the advection and diffusion terms.

\subsection{\texttt{pressure.m}}
\begin{description}
\item[Functions:] \hfill \\ \texttt{pressure} \\ \texttt{divergence} 
\end{description}
This file contains the functions needed to calculate the Poisson equation, which is solved to find the pressure in \texttt{poissonSolve.m}.

\subsection{\texttt{shear.m}}
This file contains a script that solves a shear flow problem, which is intended to demonstrate the shear flow initial condition.

\section{Acknowledgements}

Many thanks to Tina Chow and her CE200B classes in spring 2013 and 2014 for their good humor and help fixing bugs and typos.  Percy Link, Diane Taylor, David Wiersema, Bowen Zhou, Jason Simon, and Olivia Hoang all gave great suggestions for improvements.  Thanks to Ian Tse for enthusiasm, and to Joe Grimm for help with the pictures.  Special thanks to Dr. Phillip Colella for the fun shear flow case.

\end{document}








